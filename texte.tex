%st&pdflatex
% \documentclass[12pt, a4paper,oneside]{article} %article pour voir la todolist
\documentclass[12pt, a4paper,oneside]{amsart} %amsart pour faire joli
% \usepackage[T1]{fontenc} % disbaled for the compilation problem
% \usepackage[utf8]{inputenc}
% \usepackage[frenchb]{babel}
\usepackage{amssymb, amsfonts, mathrsfs}
%\usepackage{a4wide, enumerate,} %test
% \usepackage{url} 	%test pour le probleme de compilation
% \usepackage{textcomp}
\usepackage[bookmarks]{hyperref}
% \usepackage{showkeys} % for work version
% \usepackage[textwidth=1.5in,backgroundcolor=orange,linecolor=black]{todonotes}
%\usepackage[margin=2in]{geometry}
%\usepackage{a4wide}

\usepackage{lmodern}

%\usepackage{fullpage}
%\linespread{1.2}
\hfuzz=15pt

\usepackage{amsmath}
\usepackage{amsthm}
%\usepackage{amssymb}
%\usepackage{mathrsfs}
\usepackage{mathtools}  
\mathtoolsset{showonlyrefs}  

%drawing
\usepackage{tkz-euclide}
\usepackage{tikz}
\usetikzlibrary{patterns}
\usepackage{thmtools, thm-restate}
% theoreme
%\newtheorem{lemme}{Lemme}
\newtheorem{theorem}{Théorème}
\newtheorem{definition}{Définition}
\newtheorem*{theorem*}{Theorem}
\newtheorem{lemma}[theorem]{Lemme}
\newtheorem{propo}{Proposition}[section]
\theoremstyle{definition}
\newtheorem{Def}{Définition}[section]
\theoremstyle{remark}
\newtheorem{Rq}{Remarque}
\theoremstyle{remark}
\newtheorem{exam}[propo]{Exemple}
\newtheorem{rmk}[propo]{Remarque}

% theorem volé à Deroin-Tholozant

%\newtheorem{coro}[theorem]{Corollary}
%\newtheorem{defprop}[propo]{Definition-Proposition}
%\newtheorem{question}{Question}
%\newtheorem*{conj}{Conjecture}
%\theoremstyle{theorem}
%\newtheorem*{CiteThm}{Theorem}

% macro
\renewcommand\P{\mathbb{P}}
\newcommand\C{\mathbb{C}}
\newcommand\R{\mathbb{R}}
\newcommand\N{\mathbb{N}}
\newcommand\D{\mathbb{D}}
\newcommand\Z{\mathbb{Z}}
\newcommand\HH{\mathbb{H}}
\newcommand\Sp{\mathbb{S}}
\newcommand\E{\mathbb{E}}
\newcommand\T{\mathbb{T}}
\newcommand\TT{\mathcal{T}}
\newcommand\LL{\mathcal{L}}
\newcommand\BB{\mathcal{B}}
\newcommand\FF{\mathcal{F}}
\newcommand\PP{\mathcal{P}}
\newcommand\MM{\mathscr{M}}

\newcommand\CC{\mathcal{C}}
\newcommand\Co{\mathscr{C}}


\newcommand\OP{\mathbf{OP}}
\newcommand\BigO{\mathcal{O}}


\newcommand\Sup[1]{\underset{#1}{\sup}}
\newcommand\ti[1]{\widetilde{#1}}
\newcommand\tq{\; | \;}
\newcommand\Div{ {\rm div}}
%\newcommand\norme[1]{\Vert #1 \Vert_{\infty}} 
\newcommand\supp{\mathrm{supp}}
\newcommand\grad{\mathrm{grad}}
\newcommand\sh{\mathrm{sh}}
\newcommand\ch{\mathrm{ch}}
\newcommand\tr{\mathrm{tr}}
\newcommand\Vol{\mathrm{Vol}}


\newcommand\Isom{\mathrm{Isom}}
\newcommand\Conv{\mathrm{Conv}}
\newcommand\Min{\mathrm{Min}}
\newcommand{\cat}[1]{\mathrm{CAT}\left( #1 \right)}
\newcommand{\Eq}[1]{\mathrm{Eq}\left( #1 \right)}
\newcommand{\ps}[1]{\left< #1 \right>}
\newcommand{\lie}[1]{\mathfrak{#1}}



% valeur absolue
\usepackage{mathtools}
\DeclarePairedDelimiter\abs{\lvert}{\rvert}
\DeclarePairedDelimiter\norm{\lVert}{\rVert}


\title{Dynamique Homogène}
\author{La Team Caipi}

\begin{document}

\maketitle

% \listoftodos

\tableofcontents

\section{Pistes bibliographiques}
\label{sec:bibli}

Bourbaki de Ghys sur Ratner:
\cite{ghysDynamiqueFlotsUnipotentsa}.
La Takagi lecture de Benoist-Quint:
\cite{benoistIntroductionRandomWalks2012}.
Bourbaki de Ledrappier sur BQ:
\cite{ledrappierMesuresStationairesEspaces}.

\section{Introduction à la dynamique homogène}
\label{sec:intro}

\subsection{Translation sur le tore}
\label{sub:transl_tore}

On note $ \T^{d} $ le tore de dimension $ d $.
On fixe $ v \in \R^{d} $ et considère $ T $
la translation $ x \mapsto x + v $.
On a la dichotomie:
\begin{lemma}
	La translation est (topologiquement) minimal
	si et seulement si
	la famille
	$ (1, v_{1}, \dots, v_{d}) $
	est algébriquement libre sur $ \mathcal{Q} $.
\end{lemma}
\begin{proof}
	Soit on connait la classification
	des sous-groupes fermés du tore,
	soit on regarde en Fourier.
\end{proof}

On se place dans le cas minimal
(on dit que $ v $ est générique).
La dynamique de $ T $ est très particulière:
\begin{itemize}
	\item c'est la translation sur un groupe
		(compact, abélien).
	\item c'est une isométrie (drift = dérive = 0).
\end{itemize}

La translation $ T $ préserve (par définition)
la mesure de Haar $ \lambda $ sur le tore.
\begin{lemma}
	La mesure de Haar $ \lambda $
	est ergodique pour $ T $.
\end{lemma}
\begin{proof}
	On décompose une fonction $ L^{1} $ 
	invariante en Fourier
	et on voit que les coefficients
	non constant doivent être nuls.
\end{proof}

La translation est en fait uniquement
ergodique. On peut le voir de plusieurs manières:

De manière générale, sur un groupe
abélien compact, si une translation est ergodique
pour la mesure de Haar alors elle est uniquement ergodique
(voir \cite{katokIntroductionModernTheory1995} prop. 4.2.3).

Sinon méthode par "dérive nulle":
soit $ \mu $ une autre mesure ergodique.
Soit $ f $  une fonction continue
Soit $ x $ générique pour $ \lambda $
et $ y $ générique pour $ \mu $,
c'est à dire que
\begin{align}
	\frac{1}{n}
	\sum_{k=0}^{n-1}
	f(T^{k} x)
	&\to
	\int
	f
	d \lambda
	\\
	\frac{1}{n}
	\sum_{k=0}^{n-1}
	f(T^{k} y)
	&\to
	\int
	f
	d \mu.
\end{align}
Comme l'orbite de $ x $
est dense, on peut supposer
que $ x $ est aussi proche de $ y $
que l'on veut.
Plus précisemment, soit $ \varepsilon > 0 $
et $ \delta > 0 $ un module de continuité de
$ f $ pour $ \varepsilon $.
On suppose que $ d(x,y) < \delta $.

Comme $ d(T^{k}x, T^{k} y) = d(x,y) $
on a
$ 
f(T^{k}x)
=f(T^{k} y)
+ O( \varepsilon)
$ 
pour tout $ k $.
Donc:
\begin{align}
	\frac{1}{n}
	\sum_{k=0}^{n-1}
	f(T^{k} x)
	&=
	\frac{1}{n}
	\sum_{k=0}^{n-1}
	f(T^{k} y)
	+
	O(\varepsilon)
	\\
	& \implies
	\\
	\int
	f
	d \lambda
	&=
	\int
	f
	d \mu
	+
	O(\varepsilon)
	.
\end{align}
Comme c'est vrai pour tout $ \varepsilon $,
on a l'égalité.

\section{Benoist-Quint sur le tore}
\label{sec:BQ_tore}

On considère le groupe
$ SL_{d} (\Z) $
agissant sur le tore
$ \T^{d} $.
Fixons une mesure de probabilité
$ \mu $ sur $ SL_{d} (\Z) $
telle que:
\begin{itemize}
	\item le support de $ \mu $ est finie
	\item le (semi)groupe
		$ \Gamma $
		engendré par
		le support de $ \mu $
		agit
		sur $ \R^{d} $
		de manière proximal
		et
		fortement irréductible.
\end{itemize}
On s'intéresse à la marche aléatoire
engendrée par $ \mu $ et son action
sur $ \T^{d} $.
On veut comprendre le théorème
de Benoist-Quint \cite{benoistMesuresStationnairesFermes2011}
suivant:
\begin{theorem}
	Toute mesure de probabilité
	$ \mu $-stationaire sur le tore
	est une combinaison
	de la mesure de Haar sur le tore
	et d'atomes.
\end{theorem}

Plan grossier de la preuve:
\begin{itemize}
	\item On introduit l'espace des tirages $ (B,\beta) $
		et on décompose la mesure stationnaire
		$ \nu = \int \nu_{b} \beta(db) $.
		C'est un résultat classique de Furstenberg.
	\item Il existe une application $ b \mapsto V_{b} $
		qui à chaque tirage associe une droite
		de $ \R^{d} $
		telle que pour toute valeure d'adhérence (projective)
		$ \pi $ 
		de $ b_{1} \cdots b_{n} $
		on ait $ Im(\pi) = V_{b} $
		(c'est la direction de contraction de $ b_{1} \cdots b_{n} $).
		C'est aussi Furstenberg, cela résulte du point précédent
		appliqué à une mesure stationnaire sur l'espace projectif
		et la proximalité.
	\item Lemme clé: la proba $ \nu_{b} $ est $ V_{b} $-invariante
		(c'est à dire par le flot dans la direction $ V_{b} $ ?)
		Donc c'est Haar sur un sous-tore. 
	\item Lemme intermédiaire: un $ SL_{n} $-espace dénombrable
		alors toute proba $ \mu $-stationaire et ergodique
		est $ \Gamma $-invariante et de support fini.
	\item Soit $ \Phi $ l'application de $ B $
		vers $ ST $ l'ensemble des sous-tores de $ \T^{d} $
		qui à $ b $ associe
		la composante connexe du stabilisateur
		de $ \nu_{b} $.
		L'ensemble $ ST $ est dénombrable
		et est munie de la mesure stationaire
		$  m = \Phi_{*} \beta $.
		Par le lemme intermédiaire,
		le support de $ m $ est fini
		et $ \Gamma $-invariant.
	\item Si un certain sous-tore de cette ensemble
		fini n'est pas le tore en entier,
		alors on peut contredire la forte irréductibilité.
	\item Finalement, presque sûrement $ \nu_{b} $ est Haar
		sur le tore. On en déduit que $ \nu $ est Haar
		sur le tore, par $ \nu = \mathcal{E}(\nu_{b}) $.
\end{itemize}

\bibliographystyle{alpha}
\bibliography{ref}
\end{document}
