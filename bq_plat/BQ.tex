\documentclass[11pt]{article}
\usepackage[T1]{fontenc}
\usepackage{amsthm}
\usepackage{amsmath}
\usepackage{amsfonts}
\usepackage{amssymb}
\usepackage{tikz}
\usepackage{geometry}
\usepackage{stmaryrd}
%\geometry{hmargin=2.5cm,vmargin=1.5cm}

\newtheorem{thm}{Theorem}[section]
\newtheorem{dfn}{Definition}[section]
\newtheorem{prop}{Proposition}[section]
\newtheorem{cor}{Corollary}[section]
\newtheorem{lem}{Lemma}[section]
\newtheorem{rem}{Remark}[section]
\newtheorem{theorem}{Theorem}
\renewcommand*{\thetheorem}{\Alph{theorem}}
\newcommand{\supp}{\Gamma_{\mu}}
\newcommand{\Z}{\mathbb{Z}}
\newcommand{\R}{\mathbb{R}}


% \parindent=0em

\title{Stationary measures and orbit closures for the action of $SL_2(\mathbb{Z})$ on the torus following Y.Benoist \& J-F.Quint.}
\author{Florent Ygouf}

\begin{document}
 
\maketitle

\section{Promoting invariance to classification}
\label{sec:promoting_invariance_to_classification}

\subsection{Unstable leawise measures}
\label{sub:unstable_leawise_measures}

Let $ b \in B $.
Define
$ B_{n} = (b_{k})_{k\ge n}  $
and
$ B^{n} = (b_{k})_{k \le n}  $.
Define
$ \pi_{n} : B \to B_{n} $
and
$ \pi^{n} : B \to B^{n} $
the projection.
Given $ b \in B $,
we embed
$ B_{-n} $
in
$ B_{-(n+1)} $
by
$ u \mapsto b_{-(n+1)} u $.

The unstable leaf
$ W^{uu}(b) $ 
through $ b $ is
by 
defintion
the set of $ c \in B $
such that there exists
$ n \in \Z $
such that
$ \pi^{n} (b) = \pi^{n} (c) $.
It is homeomorphic to the
inductive limit
$ \lim_{n \to \infty}  B^{-n} $
using the embeding defined above.
We denote by
$ W^{uu}_{b,n} $
the fiber of $ b $
with respect to
$ \pi^{n} $,
which is homeorphic to $ B^{-n} $.

% The time $ n $ spliting of $ B $
% (analog to a foliation chart)
% is the homeomorphism
% \begin{equation}
% 	(\pi^{n}, \pi_{n+1}):
% 	B
% 	\to
% 	B^{n} \times B_{n+1}
% 	.
% \end{equation}

We want to
construct the leafwise measures
associated to the foliation
$ W^{uu} $.

Let
$ \beta_{b}^{n} $ 
be the conditional measures
of $ \beta $
with respect to the
$ \sigma $-algebra of Borelian
saturated by the fibers of
$ \pi^{n} $.
It satisfies
\begin{equation}
	\beta
	=
	\int
	\beta_{b}^{n} 
	d\beta(b).
\end{equation}
These measures satisfy
\begin{equation}
	\beta_{b}^{n-1}{| W^{uu}_{b, n} } 
	\propto
	\beta_{b}^{n}
	,
\end{equation}
hence they define a measure up to scalars
on
$ W^{uu}_{b} $
that we denote by $ \beta_{b} $.


\begin{prop}
	The family of measures
	$ (\beta_{b})  $
	satifisfies
	$ \beta_{b}{| W^{uu}_{b,n} }
	\propto
	\beta_{b}^{n} $.
\end{prop}

Now in $ B^X $.
The unstable leaf with respect to $ T $
through
$ (b,x) $ is
\begin{equation}
	W^{uu}_{(b,x)}
	=
	\left\{ 
		(c, y),
		c \in W^{uu}_{b},
		y \in x + \R v_{b}
	\right\}
	.
\end{equation}
The leafwise measures associated
to this foliation are
\begin{equation}
	\beta^{\nu}_{b,x}
	=
	\beta_{b}
	\otimes
	\nu_{b,x}
	=
	\beta_{b}
	\otimes
	Leb_{x + \R v_{b}} 
	.
\end{equation}

\begin{prop}
	$ T $ is ergodic
	with respect to
	$ \beta^{Leb} $.
\end{prop}
\begin{proof}
	It is a particular case of the following statement:
	if $ m $ is a $ \mu $-stationary measure and
	$ m $ is invariant and ergodic with respect to
	$ g $ such that $ \mu(g) > 0 $ then
	$ m $ is ergodic as a $ \mu $-stationary measure.

	Let $ m = m_{1} + m_{2} $ be a decomposition
	of $ m $ with $ m_1 $ and $ m_{2} $ $ \mu $-stationary measures.
	Without loss of generality we can assume that there exists
	a Borel set $ A $ with $ m_{1}(A) > 0 $ and
	$ m_{2}(A) = 0 $.
\end{proof}

\begin{prop}
	The basin
	$ \Omega_{\beta^{\nu}} $ 
	of the measure
	$ \beta^{\nu} $
	for $ T $
	has positive measure with respect to
	$ \beta^{Leb} $.
\end{prop}

\begin{cor}
	$ \nu = Leb $ 
\end{cor}





% \bibliographystyle{plain}
% \bibliography{biblioBQ}

\end{document}
